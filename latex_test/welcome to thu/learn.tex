\documentclass[12pt]{article}

\usepackage[UTF8]{ctex}

\usepackage{setspace}%设置行间距
\doublespacing
%2倍行距

\usepackage{mdframed}%小灰块
\usepackage{xcolor}%字体颜色
\usepackage{ulem}%删除线
\usepackage{hyperref}%超链接

\hypersetup{
    colorlinks=true,
    linkcolor=black,
    urlcolor=blue,
    citecolor=black
}

\title{自学、善用搜索和信息差}
\author{上一次更新于}
\date{\today}

\begin{document}
\maketitle
补充这一篇的目的在于向同学谈谈“自学”。
原本在上一篇《入学之后》里就计划谈谈自学,但因为实践活动和懒癌发作鸽掉了,
最后只发了阉割版,但愿这次能写得完整些————虽然已经因为桂林旅行又鸽了一次。

对于没有大学经验、缺乏具体需求的大一新生,也很有必要强调关于自学的问题。
对于我本人而言,偶尔这样总结一下,也是很有好处的。

无需再介绍学校的制度,文风也就更自由了。
%以上算是前言
\begin{itemize}
    \item 怎样自学
        \begin{itemize}
            \item 大学与高中的比较
            \item 边用边学
        \end{itemize}
    \item 善用搜索
        \begin{itemize}
            \item 搜索的技巧
            \item 不同的源
            \item 过程为例
        \end{itemize}
\end{itemize}

\section{怎样自学}
%“自学”仿佛强调着抛掉学校和老师,闭门造车、无依无靠,才算独立自主。“鹏之徙于南冥也,水击三千里”,岂是自由自在?
%独立和依靠难以划分。即使不靠学校老师,也还有参考书和网课视频的作者们支持着你,即使没有这些,你也正站在历史上的巨
%人们的肩膀上。前人栽树后人乘凉,“礼固宜然”,后人又有什么不好意思呢?300年前的知识和20年前的经验,又何必要厚此薄彼呢?

关于写文章,有人说天下文章一大“抄”字,那么做学问,也可以说有一大“借”字。
\subsection{大学与高中的比较}
在高中的范畴内,同学实践过的自学是怎样的?对我而言,包括写一本花花绿绿、考前通看一遍的错题本,
或者是高三复习时候多写张模拟卷,或是初中放假在家时打开“洋葱数学”。这些事情确实并非“老师直接布置的作业”,
做起来也常常是一个人,也许还是一个人在家里做完,看上去\textbf{孤独而高傲},配得上\textbf{顶礼膜拜}的“自学”一词。

老师就不干了:“这一届同学怎么这么不爱问问题呢?”

我不知道这是否体现出什么“青年个体化倾向”,仅从自学角度看,这是把“独立自主”看得太重了,以至于忽略了自己一直在依靠别人的事实。
不依靠学校和老师也能学好,看上去很酷,也是备受爽文偏爱的“学霸”人设的重要支持,这仿佛金庸小说中的世外高人下凡,飘飘荡荡、无依无靠,
于是本人的强力便引人浮想联翩。
\subsection{边用边学}
同学会渐渐意识到,大学的学习思维,有的确实能在高中的基础上继承扬弃,有的却只能在大学的实践中逐渐形成。我写在这里的话,也不可能
帮助你们凭空建立起应对还不存在情况的思维,只是为同学提供一点思维锻炼的材料而已。如果同学在学习实践后,还能回想起这里的话,受一
点点启发,就算是达到目的了。话说回来,大一时最接近这种情况的,大概就是写作与沟通课,这也是同学必须在大一修完的课。

写作与沟通课,旨在让大一同学初步体验学术写作的过程,在英语基础课中也有英语阅读写作A与它对应。因为大一同学的专业知识一般不足以支撑
学术写作,所以写沟课的主题限定在人文社科而非自然科学领域。写成终稿的过程漫长曲折而富有趣味性,相信同学一定会有所体会。完成长期项目
(其实写沟的任务算是短的了)的体验是难以言传的,我也只提其中的两个问题:

\textit{
1、在较长的时间跨度内,如何管理来源不一、价值不同的资料?它们可能来自电子书、中国知网收录或是浏览器的各路搜索结果。也许你见到的第一份
材料就能幸运地用到你的文稿里,但也有可能,你在希望用上它时已经找不到了。
//
2、你必然要经历周期性重复的工作,如搜索、整理、排版和对参考文献的引用和标注等,如何减少工作量呢?
}

对于信息的获取、整理和总结,同学想必经历过,

比如我发给你们的PDF文件,其实都是先用LaTex语法写好,再由VS Code编译生成的。我直到这个假期才开始慢慢学习使用LaTex,如果将这次的
PDF文件与上次作比较,同学就会发现多出了无序列表、超链接等等稍微“折腾”一些的格式。这就是边用边学。

至少对于试图自学计算机相关知识的非计算机专业学生而言,仅凭一段连续的学习是难以达到全面掌握或知其所以然的。
大一上期我们修了C语言程序基础,但其实最后学完,我也对Visual Studio 2012的大部分功能都毫无概念,也完全不明白学期初在老师指导下
配置环境有什么作用,只是模糊地知道与.cpp文件编译有关————这样说起来,我们连编译的相关过程都不太清楚,学了一个学期的编程其实也只是
当黑匣子用而已。就像我们同样不了解手机基站的工作原理,却不影响我们和家人打电话一样,
\section{善用搜索}
有时,我们在求助帖下会看见一句“善用搜索”。

\subsection{搜索的技巧}
怎样做到善用搜索呢?如果在浏览器界面上仔细查看,同学一般都能发现某些\textcolor[rgb]{0,0.39,0}{高级搜索}或是\textcolor[rgb]{0,0.4,0}{筛选}之类的功能,
用于在一般的关键词检索上额外添加检索条件,
或者如果同学研究过,还能在搜索框输出下面这种格式:

\begin{mdframed}[backgroundcolor=lightgray, linewidth=0pt]
    \textit{关键词*关键词} site:zhihu.com -site:csdn.net
\end{mdframed}

我更乐意将它们称为搜索的技巧。因为这些包括搜索表达式在内的技巧是针对某一种信息源而言的,百度能用的表达式在中国知网未必行得通,
不同搜索引擎的高级检索功能用法也不尽相同。利用浏览器插件和脚本优化搜索结果等也是类似的技巧。

学校有一门通识课《图书馆概论》或《信息素养》,会稍微涉及上述内容,同学可以考虑选课或旁听(如果设备不够可能不接受旁听)。不过这门课是有一定工作量的,
老师也常常默认你有着具体的学术检索需求(无论是学术研究还是写作与沟通课程论文),可能入门感受不好。
\subsection{学会提问}

\subsection{不同的源}
以下内容介绍我在学习生活中常使用的信息来源(或搜集信息的工具)。行文接近测评但达不到测评的体量,仅供参考。

\textbf{小红书}

常混孙吧的人对小红书的印象可能停留于部分女性用户的逆天言论。但略做筛选,小红书可以很好用。主打“实用”的小红书上包含大量指导生活具体方面的笔记(小红书发布内容的主要形式),
以“手把手教学”和“第一人称教学”为常用风格,对于经验不足者而言入门体验好。笔记中不乏大学生普遍陌生的社会素养内容,如社保相关和办事流程等,经过筛选也能比较容易地将广告式的推广
和真人经验分享区分开。实用的、需要真人经历和生活经验的内容,可以考虑在小红书上查找。

同时要注意,小红书追求“实用”的定位注定带来一些市井气(贬义),还有信息传播平台共有的问题:商业广告伪装成用户内容带来的污染。不同平台上的广告/推广各有形式,同学注意区分。

\textbf{知乎}

无论知乎现在如何做下沉市场,它的起点就注定了知乎上部分内容的专业性。近年来对下沉市场的探索将知乎上的许多内容导向娱乐和键政,而原来的一部分用户则坚持知乎的专业性,
造成现今知乎内容的割裂:一部分专业严谨,一部分娱乐至死,当然还有许多中间地带的内容,过于复杂暂且不论。
娱乐内容由同学自己把控。知乎上的专业知识内容,一般是远超出初学者水平的,完全不搭配所谓学习规律或教学计划,作为拓展阅读的材料倒不错,或者同学需要查找一些从未听说的专业术语时,
也可以在这些回答中寻找提示。而知乎上的专业者为非专业者写的文章,可以参考,不过能完全考虑初学者需求的文章,是很难得的。

\textbf{贴吧}

至少在我看来,贴吧的活跃度远不如我们提到的其他信息源,贴吧生产高质量经验贴的黄金时代在过去,现在更多的是
\sout{一群老东西和小东西翻来覆去地炒冷饭}
贴吧用户在贴吧的小团体
内讨论自己领域的热点内容。对于不入门的初学者,没什么归属感和体验感在。倒是可以寄希望于\sout{挖坟}查找过去的经验贴或者等待大佬回复你的求助贴。

\textbf{B站}

对于善于利用B站上学习资源的人而言,B站大学是再贴切不过的称呼,同学应该比较熟悉,不再赘述。

\textbf{人民日报}

(手机上有同名app,网页为\href{http://www.people.cn/}{人民网})
\subsection{过程为例}

\section{配置举例}
\end{document}