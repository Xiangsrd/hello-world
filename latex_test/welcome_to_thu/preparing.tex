\documentclass{article}
\usepackage{ctex}
\usepackage{geometry}
\usepackage{setspace}
%行间距
\usepackage{booktabs}
%三线表
\usepackage{amsmath}
%为了GPA表的那个≥号
\usepackage{url}
%脚注中的url网址要用url{}包起来

\title{欢迎来到清华大学}
\author{最后更新于}
\date{\today}

\geometry{a4paper,left=3cm,right=3cm,top=2cm,bottom=2cm}
\doublespacing

\begin{document}
\maketitle

\section{报名阶段}
随录取通知书寄来的各类小册子和独立表单(还有代扣学费的银行卡),加上到校报名时收到的学生手册等,
将为同学提供包括学校政策在内的大量阅读材料,静下心去读,注意与高中的区别。
如果同学高中就有读学生手册的习惯就更好了。
这些材料是第一手资料,同学有关于政策方面的问题都可以在这些资料中找到解答。


到校报名后,同学要处理很多报名注册入住方面的事务,凭着对那些材料的熟悉,一般不会遇到什么问题。
在这里只做一点其他方面的提示:

{
\kaishu 入学时,天猫超市会在紫荆学生公寓和紫荆操场间搭摊位,生活用品(除床上用品)和自行车都有出售。
以后也可自行前往校内商超购置(见后文资源推荐)。

不要忘记在C楼注册学生卡,这是以后每个学期开学前雷打不动的例行事项。

辅导员会联系你加入班级微信群,以后的各种联系也以微信为主。
2023级新生QQ群(学生自组织)会在报名前建好,同学可以自行加入。
2023级新生家长群(家长自组织)也可请家长自行加入。(警惕诈骗!)
}

\subsection{各种账号}
报名后,同学会了解到学校提供了非常多的网络在线服务和资源,初次注册账号会比较混乱,
建议同学每次注册和登陆都记住“网页-账号/用户名-密码-用途”,熟悉过后就无所谓了。
其中,尤其重要的包括:电子身份服务系统账号激活,图书馆读者身份激活,校园网Tsinghua-Secure网络的配置,
家园网账号注册及入住前的住宿习惯调查。


拿到录取通知书后就可以按照里面各种小册子的提示开始操作,有的步骤报名后才能完成,不要忘记向辅导员寻求帮助。


\subsection{熟悉环境}
众所周知,清华以校园面积大于北大而著称(?)。为了让新生尽快熟悉校园环境,学校将发放纸质版校园地图,
同时新生军训中将安排定向越野作为破冰活动。入学后,还不熟悉的同学可以借助手机导航
(大学的导航可以具体到教学楼,这就是高中所不能企及的了),或乘内外环校园巴士(见后文资源推荐)。

闲时与室友一起在校园中自由漫步(骑车漫游)也可作为熟悉环境/人际交往的活动。
比如和室友一起解锁一到六教/新水/旧水/系馆/文科图书馆/荷园等。

不过注意,学校不提倡使用电动车/电瓶车。紫荆学生公寓内停车场禁止电动车入内,附近也未配备充电桩,宿舍内充电插头有
功率限制。推荐同学学好自行车。校内共享单车也很方便。

\subsection{资源推荐}
学校提供了众多教学科研资源,但对于不了解的新生而言,克服信息差利用这些资源十分困难。
为了解决这个问题,一代代清华学生创造出众多工具/服务/信息集合,以下列出比较核心的几个:

Thu Services
https://thu.services/

Thu合集
https://thu.wiki/docs/intro

华清大学课程攻略共享计划
https://in.closed.social:9443/pastExam/

新T树洞 (查找经验贴及合集)
https://new-t.github.io/\textnormal{\#}\textnormal{\#}\textnormal{\#}


其中Thu Services和Thu合集本身集合了极多工具/服务,不再重复列出。在院系范围内也许也有前人建立的互助/信息集合


如 软件学院互助文档
https://ssast-readme.github.io/


感谢一代代为消除信息差无私奉献的清华同学!


\section{GPA}
\subsection{绩点是什么}
在大学,需要比较(总)成绩时不再使用总分,因为不同同学的总分很可能不同。
对于获得百分制成绩的科目,常常转化为绩点GPA来比较。清华的绩点在0-4.0之间。


\subsection{清华大学GPA对应关系}
\begin{tabular}{l|c|c|l}
    \hline
    等级制成绩&绩点&对应百分制成绩范围&备注\\
    \hline
    A+&4.0&95-100& \\
    A&4.0&95-100& \\
    A-&4.0&90-94& \\
    B+&3.6&85-89& \\
    B&3.3&80-84& \\
    B-&3.0&77-79& \\
    C+&2.6&73-76& \\
    C&2.3&70-72& \\
    C-&2.0&67-69& \\
    D+&1.6&63-66& \\
    D&1.3&60-62& \\
    F&0&0-59& \\
    P&不参与计算&$\geq60$&记P/F\\
    F&0,参与计算&0-59&记P/F\\
    \hline
\end{tabular}
\bigskip

%这样才能连空两行
此为2018参考版,最新版请查阅学生手册或咨询辅导员。

具体课程给绩点的方式可能不同,如限制前50\textnormal{\%}评A或对A+有额外要求。

总GPA即为各课程GPA的加权平均,“权”为课程学分。

保研外推/评奖评优/专业分流/转专业均需参考GPA。

\subsection{任选课\textnormal{\&}体育课}
选课系统中,课程按照选课属性分为必修、限选、任选,由院系培养方案中每学期的推荐课表决定;
院系培养方案中,课程按照课程属性分为必修、限选、任选。

大一年级基本上是四年中推荐课程学分最多的,尽管同学刚刚入学本不应该这样安排。
在大一第一学期并不推荐同学选任何选修课。尽管舆论与学校的改革都在呼唤复合的、交叉方向的、综合性的人才,
但这毕竟不是在大学多开几门通识选修课就能解决的事。。。。。。。

清华同学每个学期有一次选择某课成绩记P/F(即通过后不参与GPA计算,只要拿及格分就行)的机会,
在一般的院系政策中,只允许对课程属性为任选的课进行这项操作(具体咨询院系教务老师)。也有同学不记P/F,
用任选课的GPA提高自己的总GPA,这点见仁见智,保研时考虑GPA一般只看必修和限选课。

另,清华挂科的课直接重修,没有补考,且计算总绩点时,不及格的成绩不会被重修成绩覆盖,这是非常亏的,
所以哪怕学不下去也请不要挂科\footnote{全文关于政策措施规定的介绍,请以最新的学生手册和辅导员通知为准。}
。
“为祖国健康工作五十年”的口号近年来已经进步到了“争取至少为祖国健康工作五十年”。
体育课成绩由体质测试成绩、体育专项成绩和课外体育锻炼成绩组成,体育课实行等级制成绩,记1学分,
实际按两学分上课,只要总分及格即可。

\section{学习\textnormal{\&}资源}
\subsection{学习资源(教学\textnormal{\&}自习)}
同学入学后,会收到本系的教学计划/培养方案,里面推荐了各学期的选课课表和课程的前后依赖关系,
后面学期的选课都要参考教学计划里的信息。对于同一届学生,教学计划不会反复变动,
但不同届的学生,适用的教学计划经常有变化。

教材上,清华很多课程用的是学校老师参与编写的教材,基本上是适合教学进度的,不论是不是适合初学者。
另外,大一线性代数的教材,只要是中文的,都无一例外地烂,初学者很难搞懂线性代数在搞什么,
可以借助B站宋浩老师等的视频预习一下。但要注意,宋浩老师的视频和同济教材是达不到清华难度要求的,
高难度部分(常常涉及理论证明)还请经常请教老师、助教、答疑坊。

老师在第一节课将公布open office hour,在这个时间段内同学可以前往老师办公室请教习题或课内课外内容。
其他时间可以请教课程助教。答疑坊内容请参考“乐学”公众号及其推送
\footnote{\url{https://mp.weixin.qq.com/s/BLnF_9wM24bNxVq1uvO3aQ}}
。
常见形式包括微信小程序“清答疑”、微信引导群及每晚六教某教室的志愿者线下答疑。

其他资源见前文资源推荐。

进入大学,不太可能像高中一样组织集体自习课了。
个人自习常常在空教室或图书馆,C楼和清华学堂也可能有位置,不推荐在寝室自习。

\subsection{课外活动}
清华社团、协会多,由社团、学校党委团委、某些院系和其他机关组织的课外活动也多,
在学堂路两侧常有宣传海报,在微信公众号也会同步更新活动通知。志愿者等活动同理。

按照清华的寝室管理办法,同学理论上可以夜不归宿(连续三天不刷宿舍门禁会上报院系),非上课时间都可以
自由出校游玩,自己把握。

\section{更多内容}
同学可以参考知乎《上海交通大学生存手册》。

\end{document}